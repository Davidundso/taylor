%%%%%%%%%%%%%%%%%%%%%%%%%%%%%%%%%%
% HTWK Leipzig LaTeX Vorlage - Abschlussarbeit
% Interne Referenznummer 3_1 Overleaf
%
% Die Vorlage orientiert sich an der "Richtlinie für die Anfertigung von wissenschaftlichen Arbeiten" der Fakultät Ingenieurwissenschaften.
% Es besteht kein Anspruch auf Richtigkeit.
% 
% Das Dokument basiert teilweise auf den Vorlagen von :
% Markus Voerkel
% https://de.overleaf.com/latex/templates/thesis-template-for-hochschule-fur-technik-wirtschaft-und-kultur-leipzig/nqpftcjmmtts
%
% Linda Vogel & Jon Arnt Kårstad
% https://www.overleaf.com/latex/templates/template-projekt-htwk/mphqwccfvvwy
%%%%%%%%%%%%%%%%%%%%%%%%%%%%%%%%%%
%
% Veröffentlicht unter CC BY-SA 4.0 - Lizenz
%
%%%%%%%%%%%%%%%%%%%%%%%%%%%%%%%%%%

%---------------------------------------
%	Dokumentenklasse
%---------------------------------------

% Dokumentenklasse %
\documentclass[
ngerman,        
toc=flat,
toc=chapterentrywithdots,
captions=tableabove,
listof=entryprefix,
listof=leveldown,
fontsize=12pt,
numbers=noenddot]
{scrreprt} % sccrreprt ist eine Klasse die für Naturwiss. Arbeiten geeignet ist (sehr flexibel) %

\usepackage{comment}
\begin{comment}


 Optionen:

ngerman: Setzt die Sprache auf Deutsch. Hierbei werden deutsche Silbentrennung, Datumsformate und andere sprachspezifische Formatierungen angewendet.

toc=flat: Das Inhaltsverzeichnis (toc für "table of contents") wird flach dargestellt, d.h., ohne Einrückung der Ebenen.

toc=chapterentrywithdots: Die Kapitel im Inhaltsverzeichnis erhalten eine Linie (mit Punkten) bis zur Seitenzahl.

captions=tableabove: Setzt die Bildunterschriften für Tabellen oberhalb der Tabellen, was in wissenschaftlichen Arbeiten häufig üblich ist.

listof=entryprefix: Fügt einen Präfix für die Einträge in den Abbildungsverzeichnissen hinzu.

listof=leveldown: Die Hierarchie der Verzeichnisse wird eine Stufe heruntergesetzt, z.B. für detailliertere Strukturen.

fontsize=12pt: Legt die Schriftgröße im gesamten Dokument auf 12pt fest. Dies ist eine übliche Größe für wissenschaftliche Arbeiten.

numbers=noenddot: Entfernt den Punkt am Ende von Kapitel- und Abschnittsnummern (z.B. „1“ statt „1.“)

\end{comment}
%---------------------------------------
%	Laden von Paketen
%---------------------------------------

\usepackage{babel}  
% sorgt dafür, dass die Sprache korrekt behandelt wird, also Silbentrennung, Anführungszeichen
\usepackage{lmodern} %font
\usepackage[T1]{fontenc} % umlaute äö
\usepackage{float}   % Platzierung von Tabellen: mit [H] genau dahin wo sie sind
\usepackage{ragged2e}  % Text bündig platzieren

% Geoemetry %
\usepackage{geometry}  
\geometry{
	top = 25mm,
	headsep = 12mm,  % Kopfzeile und text
	left = 32mm,  % mehr wegen binden später 
	right = 24mm, 
	bottom = 25mm,
	}

\usepackage[letterspace=150]{microtype}   % Anpassungen wie Zeilenumbruch, Buchstabenabstände und Silbentrennung
\usepackage[onehalfspacing]{setspace}  % spacing im text


% Caption %
\usepackage[labelfont={bf,sf},font={bf}, labelsep=space, singlelinecheck=off]{caption} 
\captionsetup[figure]{justification=centering} % Bildunterschriften (figure) zentriert dargestellt
\captionsetup[table]{justification=raggedright} % tabellenunterschr linksbündig

\begin{comment}
labelfont={bf,sf}: Das Label (also der Teil der Beschriftung, der "Abbildung" oder "Tabelle" anzeigt) wird fett (bf = boldface) und serifenlos (sf = sans-serif) dargestellt.

font={bf}: Der gesamte Text der Caption wird fettgedruckt dargestellt.

labelsep=space: Zwischen dem Label (z. B. „Abbildung 1“) und der eigentlichen Beschreibung wird ein einfacher Abstand (space) eingefügt. Standardmäßig gibt es hier oft einen Doppelpunkt oder Punkt, der durch space ersetzt wird.

singlelinecheck=off: Diese Option verhindert, dass die Caption automatisch zentriert wird, wenn sie nur eine Zeile lang ist. Normalerweise werden einzeilige Beschriftungen zentriert; hier bleibt aber die Linksbündigkeit erhalten (außer anders angegeben).

\end{comment}



% Bibtex %
% biber = Prozessor; numeric-comp = Zitate als Zahlen und [1-3] statt [1], [2], [3]
% sorting=none: Die Einträge im Literaturverzeichnis werden in der Reihenfolge aufgelistet, in der sie im Text zitiert werden, anstatt alphabetisch oder nach anderen Kriterien sortiert zu werden.
% Zitationsnummern erst am Ende des Dokuments Zuweisen
\usepackage[
    backend=biber,
    style=numeric-comp,
    sorting=none,
    defernumbers=true
]{biblatex}
\addbibresource{Literatur.bib}  % quelle der citations

\usepackage{csquotes}  % für zitate (wörtliche Reden)
\usepackage{amsmath}  % mathem. Notationen
\usepackage{graphicx} % Enfügen von Graphiken

% Vorbereitung der Verzeichnisse %
\usepackage[nohyperlinks,printonlyused]{acronym}  % abkürzungen haben keinen Link, nur Abk zeigen die benutzt werden

\begin{comment}
colorlinks=true: Aktiviert die farbige Darstellung von Links im Dokument. Anstelle von standardmäßigen umrandeten Boxen werden die Links in verschiedenen Farben angezeigt.
linkcolor=black: Legt fest, dass interne Links (z. B. zu Kapiteln oder Abschnitten) schwarz dargestellt werden.
urlcolor=blue: URLs werden in blauer Farbe angezeigt. Dies sorgt dafür, dass Hyperlinks zu Webseiten klar erkennbar sind.
citecolor=black: Zitate (z. B. aus dem Literaturverzeichnis) werden ebenfalls schwarz angezeigt, was eine einheitliche Darstellung mit dem Fließtext ermöglicht.
\end{comment}
% Hyperref %
\usepackage{hyperref}
\hypersetup{ 
    colorlinks=true,
    linkcolor=black,
    urlcolor=blue,
    citecolor=black}
 
% Einfügen von PDF-Dateien %
\usepackage{pdfpages}

% Einfacher Umgang mit Einheiten %
\usepackage{siunitx}
\sisetup{
    locale=DE,  % deutsche einheiten (mit komma)
    per-mode=fraction,  % brüche verwenden
}

% Bessere Kompatibilität der Dokumentenklasse mit div. Paketen %
\usepackage{scrhack}

%---------------------------------------
%	Weitere Konfigurationen
%---------------------------------------

%Kopfzeile mit Seitenzahl
\usepackage[autooneside=false]{scrlayer-scrpage}
\clearpairofpagestyles
\setkomafont{pageheadfoot}{\footnotesize}
\chead{{--~}\pagemark{~--}}


%Konfiguration Verzeichnisse
\BeforeStartingTOC[toc]{\singlespacing} 
\BeforeStartingTOC[lot]{\renewcommand\autodot{:}}
\BeforeStartingTOC[lof]{\renewcommand\autodot{:}}


%Anpassung der Kapitelüberschrift
\renewcommand*{\chapterpagestyle}{scrheadings}
\RedeclareSectionCommand[%
beforeskip=0pt,
afterskip=30pt,
afterindent = false,
font=\Huge\bfseries\rmfamily % \rmfamily stellt sicher, dass eine Serifenschrift genutzt wird
]{chapter}

%Verhindere Zeilenumbruch innerhalb \cite[Seite]{quelle}
\renewcommand*{\prenotedelim}{\addnbspace}
\renewcommand*{\postnotedelim}{\addcomma\addnbspace}
\renewcommand*{\multicitedelim}{\addcomma\addnbspace}
\renewcommand*{\extpostnotedelim}{\addnbspace}
\renewcommand*{\volcitedelim}{\addcomma\addnbspace}


%%%%%%%%%%%%%%%%%%%%%%%%%%%%%%%%%%%%%%%%%%%%%%%%%
%
%       Bitte hier die eigenen Daten eingeben
%       für Generierung der Titelseite etc.
%
%%%%%%%%%%%%%%%%%%%%%%%%%%%%%%%%%%%%%%%%%%%%%%%%%

\newcommand{\autor}{David Suckrow} % Vorname Name
\newcommand{\mnr}{12345} % Matrikelnummer, bspw.: 12345

\newcommand{\betreuerI}{Prof. Dr.-Ing. Moritz Musterprof} % Name 1. Betreuer bzw. betreuender Prof

\newcommand{\betreuerII}{Dipl.-Ing. Manuela Musteringenieurin} % Name 2. Betreuer
\newcommand{\betreuerIItaetigkeit}{Betriebliche Betreuerin} % Tätigkeit des Betreuers: Fachbetreuer, Wissenschaftlicher Betreuer, Betrieblicher Betreuer

\newcommand{\abschluss}{Bachelor} % Art des Studiengangs/Abschlusses: Bachelor oder Master
\newcommand{\art}{Bachelorarbeit} % Art der Arbeit: Projektarbeit, Bachelorarbeit, Masterarbeit

\newcommand{\zeitraum}{Oktober 2024 -- Februar 2025} %Monat (JJJJ) -- Monat JJJJ

\newcommand{\titel}{Curvature-based step-sizes in deep \\ Neural Networks} % Titel der Arbeit

\newcommand{\gebdatum}{01.\,01.\,2000} % Geb.datum, TT.\,MM.\,JJJJ

\newcommand{\gebort}{Leipzig} % Geburtsort

\newcommand{\datum}{01.\,12.\,2025} % z.B. Abgabedatum

%Fakultäten, nicht ändern!
\newcommand{\FAS}{Architektur und Sozialwissenschaften}
\newcommand{\FB}{Bauwesen}
\newcommand{\FING}{Mathematisch-Naturwissenschaftliche Fakultät}
\newcommand{\FDIT}{Digitale Transformation}
\newcommand{\FIM}{Informatik und Medien}
\newcommand{\FWW}{Wirtschaftswissenschaft und Wirtschaftsingenieurwesen}

% Auswahl Fakultät:
\newcommand{\fak}{\FING} % Eingabe der Fakultät (siehe oben)

\newcommand{\studiengang}{Kognitionswissenschaft} %bspw.: Maschinenbau 

%%%%%%%%%%%%%%%%%%%%%%%%%%%%%%%%%%
% Konfiguration des Anhangsverzeichnis
%
% weitere Informationen: https://komascript.de/comment/5578#comment-5578, mit Anpassungen
% Beispiel: https://komascript.de/comment/5609#comment-5609
%
%%%%%%%%%%%%%%%%%%%%%%%%%%%%%%%%%%

\DeclareNewTOC[%
  owner=\jobname,
  listname={Anhangsverzeichnis},% Titel des Verzeichnisses
]{atoc}% Dateierweiterung (a=appendix, toc=table of contents)
\DeclareNewTOC[%
  listname={Abbildungen im Anhang},% Titel des Verzeichnisses
  name=\noexpand\listoflofentryname,
]{alof}% Dateierweiterung (a=appendix, lof=list of figures)
\DeclareNewTOC[%
  listname={Tabellen im Anhang},% Titel des Verzeichnisses
  name=\noexpand\listoflotentryname
]{alot}% Dateierweiterung (a=appendix, lot=list of tables)

\makeatletter
\newcommand*{\useappendixtocs}{%
  \renewcommand*{\ext@toc}{atoc}%
  \scr@ifundefinedorrelax{hypersetup}{}{% damit es auch ohne hyperref funktioniert
    \hypersetup{bookmarkstype=atoc}%
  }%
  \renewcommand*{\ext@figure}{alof}%
  \renewcommand*{\ext@table}{alot}%
}
\newcommand*{\usestandardtocs}{%
  \renewcommand*{\ext@toc}{toc}%
  \scr@ifundefinedorrelax{hypersetup}{}{% damit es auch ohne hyperref funktioniert
    \hypersetup{bookmarkstype=toc}%
  }%
  \renewcommand*{\ext@figure}{lof}%
  \renewcommand*{\ext@table}{lot}%
}
\scr@ifundefinedorrelax{ext@toc}{%
  \newcommand*{\ext@toc}{toc}
  \renewcommand{\addtocentrydefault}[3]{%
    \expandafter\tocbasic@addxcontentsline\expandafter{\ext@toc}{#1}{#2}{#3}%
  }
}{}
\makeatother
 
\usepackage{xpatch}
\xapptocmd\appendix{%
  \addpart{\appendixname}
  \useappendixtocs
  \listofatocs

}{}{}

\BeforeStartingTOC[atoc]{\singlespacing} 
\BeforeStartingTOC[alot]{\singlespacing\renewcommand\autodot{:}}
\BeforeStartingTOC[alof]{\singlespacing\renewcommand\autodot{:}}%



%Ebenen im Inhaltsverzeichnis
\newcommand{\nocontentsline}[3]{}
\newcommand{\tocless}[2]{\bgroup\let\addcontentsline=\nocontentsline#1{#2}\egroup}

%Verwendung normaler "Gänsefüßchen"
\MakeOuterQuote{"}

%Kein Einzug nach Absatz
\setlength\parindent{0pt}

%Schriftgröße in Tabellen gleich der Schriftgröße im Dokument
\usepackage{etoolbox}
\AtBeginEnvironment{table}{\sffamily}

%%%%%%%%%%%%%%%%%%%%%%%%%%%%%%%%%%%%%%%%%%%%%%%%%%%%%%%%%%%%%%%%%


%%%%%%%%%% Platz für weitere Pakete %%%%%%%%%%%%%%
%2 Bilder nebeneinander
\usepackage{subcaption}






%%%%%%%%%%%%%%%%%%%%%%%%%%%%%%%%%%%%%%%%%%%%%%%%%%

%-------------------------------------------------
%	Beginn des Dokuments und Einbinden der Kapitel
%-------------------------------------------------

\begin{document}

\begin{titlepage}
{\centering
{\Large \textbf{Eberhard Karls Universität Tübingen}\par}
\vspace{1.75cm}
{\large \textbf{ \fak} \par}
{\large \abschluss \ \studiengang\par}
\vspace{1.75cm}
{\Large \textbf{\titel}\par}
\vspace{1.5cm}
{\large \art \par}
\vspace{3cm}
{\large  von\par}
\vspace{1.25cm}
{\autor\\[3ex]
geb. am \gebdatum\\[3ex]
in \gebort\\[3ex]
\mnr\par}}
\vfill
{\noindent Erstgutachter \betreuerI 
\par \vspace{0.25cm}
Zweitgutachter: \betreuerII
\par \vspace{0.75cm}
Tübingen, \zeitraum} % ggf. Ort anpassen
\end{titlepage}

%Aus dem Opal herunterladen, Einbinden durch einkommentieren:
%\includepdf{Inhalte/PDFs/Sperrvermerk.pdf}

%Aufgabenstellung, Einbinden durch einkommentieren:
%\includepdf{Inhalte/PDFs/Aufgabenstellung.pdf} 

\thispagestyle{empty}
\begin{center}
\large \lsstyle Erklärung
\end{center}
\vspace{1.5cm}
{\doublespacing
Ich versichere wahrheitsgemäß, die \art{} selbständig angefertigt, alle benutzten Hilfsmittel vollständig und genau angegeben und alles kenntlich gemacht zu haben, was aus Arbeiten anderer unverändert oder mit Abänderungen entnommen wurde.}\par
\vspace{2cm}
\noindent
% Autor
\begin{minipage}[t]{6.5cm}
% gepunktete Linie
\dotfill \par
% Text unter der Linie
\onehalfspacing
\autor \par
Leipzig, den \datum % ggf. Ort anpassen
\end{minipage}\par
\vspace{2cm}
\clearpage

%Danksagung, Einbinden der auzufüllenden Danksagung.tex durch einkommentieren:
%\thispagestyle{empty}
\begin{center}
\large \lsstyle
Danksagung
\end{center}
\vspace{1.5cm}
Ich bedanke mich bei meinem Betreuer \betreuerI{}. Auch \betreuerII{} möchte ich für ihre Unterstützung danken...

\tableofcontents
\tocless\addchap{Abbildungs- und Tabellenverzeichnis}
\listoffigures
\pagebreak %Nach Belieben entfernen
\listoftables

%%%%%%Nutzung des Abkürzungs- und Formelzeichenverzeichnis%%%%%%%

%Für Nutzung des Abkürzungsverzeichnis folgendes auskommentieren (Prozentzeichen entfernen):

%%%%%%%%%%%%%%%%%%%%%%%%%%%%%%%%%%%
% Die Verwendung dieses Verzeichnisses ist optional!
%
% Es ist standardmäßig deaktiviert.
% Zum Aktivieren den Hinweis in der main.tex beachten.
%
% Weitere Informationen: https://www.namsu.de/Extra/pakete/Acronym.pdf
%
% Es muss immer mindestens ein Eintrag aus dem Verzeichnis im Text aufgerufen werden!
% ggf. mit Label-Prefixen arbeiten (z.B.: \abb{ak:ASTM})
%
% Konfiguration %

\newcommand{\abb}[3]{\acro{#1}[{#2}]{#3}}

\tocless\addchap{Abkürzungsverzeichnis}

%%%%%%%%%%Verzeichnis%%%%%%%%%%%%%

\begin{acronym}[LONGESTLONGEST]
	%\abb{Label}{Abkürzung}{Langfassung}
	\abb{astm}{ASTM}{American Society for Testing and Materials}
	\abb{ttt}{TTT-Diagramm}{Time-Temperature-Transition Diagramm}
\end{acronym}

%%%%%%%%%%%%%%%%%%%%%%%%%%%%%%%%%%

% Aufruf des Kürzels im Text mit \acs{Label}
% Aufruf der Langfassung im Text mit \acl{Label}

%Für Nutzung des Formelverzeichnis folgendes auskommentieren (Prozentzeichen entfernen):

%%%%%%%%%%%%%%%%%%%%%%%%%%%%%%%%%%%
% Die Verwendung dieses Verzeichnisses ist optional!
%
% Es ist standardmäßig deaktiviert.
% Zum Aktivieren den Hinweis in der main.tex beachten.
%
% Es muss immer mindestens ein Eintrag aus dem Verzeichnis im Text aufgerufen werden!
% ggf. mit Label-Prefixen arbeiten (z.B.: \abb{fz:obfl})
%
% Konfiguration %

\newcommand{\form}[4]{\acro{#1}[\ensuremath{#2}]{#4{\acroextra{\hspace{5em}[#3]}}}}

\renewcommand*{\aclabelfont}[1]{\textbf{\boldmath\acsfont{#1}}}

\tocless\addchap{Formelverzeichnis}

\newcommand{\acls}[1]{\acl{#1} \acs{#1}}

%%%%%%%%%%Verzeichnis%%%%%%%%%%%%%
%

\begin{acronym}[LONGEST]
	%\form{Label}{Formelzeichen}{Einheit}{Beschreibung}
	\form{obfl}{A}{\si{\m^2}}{Oberfläche}
    \form{vol}{V}{\si{\m^3}}{Volumen}
\end{acronym}

%
%%%%%%%%%%%%%%%%%%%%%%%%%%%%%%%%%%


% Aufruf des Formelzeichens im Text mit \acs{Label}
% Aufruf der Beschreibung im Text mit \acl{Label}
% Aufruf von Beschreibung+Formelzeichen mit \acls{Label}


%%%%%%%%%%%%%%%%%%%%%%%%%%%%%%%%%%%%%%%%%%%%%%%%%%%%%%%%%%%%%%%%%

\chapter{Einleitung}

Artificial neural networks (ANNs) have proven to be a useful tool for complex questions that involve large
amounts of data. Our use case of predicting soil maps with ANNs is in high demand by government agencies,
construction companies, or farmers, given cost and time intensive field work. However, there are two main
challenges when applying ANNs. In their most common form, deep learning algorithms do not provide interpretable
predictive uncertainty. This means that properties of an ANN such as the certainty and plausibility of the
predicted variables, rely on the interpretation by experts rather than being quantified by evaluation metrics
validating the ANNs. Further, these algorithms have shown a high confidence in their predictions in areas
geographically distant from the training area or areas sparsely covered by training data. To tackle these challenges,
we use the Bayesian deep learning approach “last-layer Laplace approximation”
, which is specifically
designed to quantify uncertainty into deep networks, in our explorative study on soil classification. It corrects the
overconfident areas without reducing the accuracy of the predictions, giving us a more realistic uncertainty
expression of the model’s prediction. In our study area in southern Germany, we subdivide the soils into soil
* Corresponding author at: Department of Geoscience, University of Tübingen, Rümelinstraße 19-23, Tübingen 72070, Baden-Württemberg, Germany.
E-mail addresses: kerstin.rau@uni-tuebingen.de (K. Rau), katharina.eggensperger@uni-tuebingen.de (K. Eggensperger), f.schneider@uni-tuebingen.de
(F. Schneider), philipp.hennig@uni-tuebingen.de (P. Hennig), thomas.scholten@uni-tuebingen.de (T. Scholten).
Contents lists available at ScienceDirect
Science of the Total Environment
journal homepage: www.elsevier.com/locate/scitotenv
https://doi.org/10.1016/j.scitotenv.2024.173720
Received 2 February 2024; Received in revised form 31 May 2024; Accepted 31 May 2024
Science of the Total Environment 944 (2024) 173720
2
regions and as a test case we explicitly exclude two soil regions in the training area but include these regions in
the prediction. Our results emphasize the need for uncertainty measurement to obtain more reliable and interpretable
results of ANNs, especially for regions far away from the training area. Moreover, the knowledge gained
from this research addresses the problem of overconfidence of ANNs and provides valuable information on the
predictability of soil types and the identification of knowledge gaps. By analyzing regions where the model has
limited data support and, consequently, high uncertainty, stakeholders can recognize the areas that require more
data collection efforts.
1. Introduction
The use of machine learning in science has become incredibly
valuable and has significantly transformed many areas of research. The
number of studies in which methods from the field of machine learning
(ML) are used is constantly increasing (Zhang et al., 2022). Soil science
is one of the pioneers here, where extensive applications in the field of
soil mapping were already developed at the beginning of this century
(Behrens et al., 2005; McBratney et al., 2003). Today, digital soil mapping
is one of the largest areas in which the methods are widely used for
all kinds of climatic and geomorphometric regions of the World and in
different areas of soil science, which has been demonstrated by
numerous papers (Minasny and McBratney, 2016; Rentschler et al.,
2022; Scull et al., 2003; Taghizadeh-Mehrjardi et al., 2021b; Zhang
et al., 2022). Methodologically, applications of ML in soil science range
from linear regression to modelling soil properties and their relationships
to complex deep learning methods (Moore et al., 1993; Veres et al.,
2015). The increasing use of these methods is not only due to their
suitability for soil scientific and geographical questions, but also because
producing soil type maps in the traditional way with cartographers
surveying the landscape is very costly and time-consuming. This effort
can be reduced with machine learning, especially for larger or even
difficult to access areas (Behrens et al., 2005; Grunwald et al., 2011;
Hewitt, 1993). At the same time, machine learning methods and their
source code are becoming more accessible due to open-source software
and widely available computational resources (Dramsch, 2020), and
with the publication of several large open source datasets containing
digital elevation models, climate data and other remote sensing data,
especially those describing the vegetation, it is getting more convenient
to apply them (Gascon et al., 2017; McBratney et al., 2003).
Looking at the properties and functions of soils, for example carbon
and water storage and plant nutrition, the soil type as a highly integrated
prediction variable has the advantage that we can infer mechanical

results being rich in information, a major drawback of the predicted soil
maps, and especially of the survey maps, is that they do not quantify the
uncertainty of the individual soil types at a given geographical location
(Hengl et al., 2017). Instead, mostly is only given an overall accuracy
statement in the form of a single statistical number. This is usually
calculated as a coefficient using cross-validation techniques, where a
subset of the training dataset is used to quantify the uncertainty of the
overall performance (Wadoux et al., 2020). However, this is not sufficient,
especially for regional or global tasks using unbalanced data sets,
and that further analysis on uncertainty statements is needed, which was
highlighted by (Meyer and Pebesma, 2022). Also, studies considering
the uncertainty of predicted classes, like soil or vegetation classes, only
looking at the probability of the predicted class or its confidence interval,
have been criticized as well (Wadoux et al., 2020). They reported
that out of 175 papers, only 30 % included uncertainty quantification,
most were focused on achieving high prediction accuracy and only a
handful used machine learning methods for the uncertainty quantification.
It is obvious that a better understanding and quantification of the
uncertainty of soil maps modelled with ML is needed, especially when
extrapolating from the training domain or when transferring the model
to other more or less similar domains. In particular, working with ANNs
as a black box requires such an assessment, as this model class is also
known to be overconfident (Breiman, 2001; Nguyen et al., 2015; Hein
et al., 2019). This means that ANNs can predict very reliable results, in
our case soil classes, with a probability of up to 100 %, even if the input
data is incorrect or uncertain. The lack of uncertainty measurement by
the ANNs themselves makes it difficult to assess the reliability of the
model predictions, which can lead to misinterpretations and incorrect
decisions (Guo et al., 2017). With this study, we apply an ANN that
predicts soil types inside and outside the known training domain in a
trial study. We quantify the uncertainty of our model at every pixel in
the area using last-layer Laplace Approximation (LLLA) (Kristiadi et al.,
2020). Our aim is to add this uncertainty measurement to a soil classification
problem to identify and correct the overconfidence of ANNs and
to be able to spatially analyse and interpret in a following step the
prediction of the ANN and its uncertainty derived from the LLLA.
Further, we will discuss the transferability of the ANN to adjacent
similar areas. Overall, our analyses will help to better understand and
interpret results from ML models in soil science to provide new insights
into soil processes and the spatial structure of the different domains.


\begin{figure}[h!]  % h! bedeutet, dass die Grafik an dieser Stelle erscheinen soll
    \centering      % zentriert die Grafik
    \includegraphics[width=1\textwidth]{/Users/davidsuckrow/Desktop/plot.pdf}  % Pfad zur Grafik
    \label{fig:example}  % Label für Referenzierung im Text
\end{figure}
2.1
Science of the Total Environment 944 (2024) 173720
3
similarities due to the likewise terrestrial geologic formation including
sandstone. This great difference in such a small area naturally influences
the vegetation and the processes in the soil, including soil formation. In
total, the area comprises five major soil landscapes with different
characterization, shown in Fig. 1B. These are areas in which, under
similar geological, morphological and climatic conditions and under the
influence of human, a landscape-typical association of soils has
developed.
2.2. Data
Fig. 2A shows the soil types in the study area, with each number
representing a soil type and its characterization. The soil types are
determined according to the German soil classification system, which is
based on the processes taking place in the soil and their properties
(Eckelmann et al., 2005). In our area, there are 40 different soil types
and the urban area, which is represented by the number 0. A detailed
description can be seen in the Table 1, including the translation from the
German into the World Reference Base (WRB) soil systematics (WRB,
2022).
In order to preserve the diversity that is lost in this translation, we
will stick to the German classification. The soil type map used for our
prediction variable was initially provided by Landesamt für Geologie,
Rohstoffe und Bergbau (LGRB) Baden-Württemberg as a polygon map
(Fig. 2A). We converted this polygon map to a raster file using a rasterization
function based on the digital elevation grid. While the original
scale of the map is 1:50,000, its rasterization allowed to produce a raster
with pixels of 10 ×10 m. As covariates for the neural network, exemplified
in Fig. 2B, we looked for spatially dense data over the whole
region to get as detailed data as possible, which is also important for the
performance of the neural network. For this purpose, we use a digital
elevation model (Fig. 2B(a)), which was also provided by the LGRB with
a resolution of 10 m, based on which topographic indices were calculated,
also with a resolution of 10 m. The decision on which of the
variables we use as covariates is based on expert geographical knowledge
of the region, commonly used variables in the geosciences and by
using the SCORPAN model introduced by McBratney et al. (2003),
which is based on Jenny (1983). To cover most of the covariates presented
in the SCORPAN model, we also included satellite data. Copernicus
provides the Sentinel-2 data, available from 2017 in 13 spectral
bands with a 5-day repetition frequency. For us, the most important
variables are the visible (R, G, B) and near-infrared bands, which have a
resolution of 10 m. We use these spectral bands to calculate important
indices such as the Normalized Difference Vegetation Index (Fig. 2B(d))
to describe vegetation cover. Finally, we calculate the median value for
each index over the time series from March to May 2019. In our analysis,
we used the median as the mean over years to mitigate the influence of
outliers and to ensure a more robust representation of the data. To
capture the influence of geology, we add a geological map with the scale
of 1:50,000, provided by the LGRB and rasterized in the same way as the
soil type map. We provide an overview of all the covariates used for the
ANN and the corresponding references in Table 2.


\begin{figure}[h!]  % h! bedeutet, dass die Grafik an dieser Stelle erscheinen soll
    \centering      % zentriert die Grafik
    \includegraphics[width=.9\textwidth]{/Users/davidsuckrow/Desktop/plot.pdf}  % Pfad zur Grafik
    \label{fig:example}  % Label für Referenzierung im Text
\end{figure}


2.3. Model architecture
The origin of Artificial Neural Networks (ANNs) lies in the field of
image recognition, especially in the area of classification (Goodfellow
et al., 2016). These models are known for their ability to model multiple
outcomes quickly and efficiently with a large amount of data, even with
absence of prior knowledge about the data. Inspired by the neuronal
structure, they look for dependencies and patterns in the given data that
include input variables and a responding output variable. ANNS are
organized in layers consisting of neurons using a (non-)linear activation
function to transform and forward their inputs to the next layer,
allowing the ANN to learn complex patterns. The input layer receives the
input data and consists of one neuron per input feature, in our case, one
neuron per covariate. The neurons in the hidden layers pass the
weighted sum of the outputs from the previous layer to their activation

\begin{figure}[h!]  % h! bedeutet, dass die Grafik an dieser Stelle erscheinen soll
    \centering      % zentriert die Grafik
    \includegraphics[scale = 1]{/Users/davidsuckrow/Desktop/plot.pdf}  % Pfad zur Grafik
    \label{fig:example}  % Label für Referenzierung im Text
\end{figure}

function. The final layer outputs the prediction and consists of one
neuron per output variable, in our case, one neuron per soil type. During
training the weights of the connections between the layers are learned
via stochastic gradient descent to minimize a loss function measuring
the error of the predictions. There is a wide variability of different
constructs for an ANN for computation or information processing in
terms of the architecture of the neural network, the number, types and
dimensions of layers, or the activation function chosen. Since the focus
of our study is on uncertainty of machine learning models in a soil
context rather than on model performance, the simplicity of the model
was very important to us. We choose a fully connected multilayer perceptron
as described in Table 3. As the activation function for the hidden
layer, the rectified linear unit function was chosen, first used by Hahnloser
et al. (2000) and defined as 



\chapter{Grundlagen}
\chapter{Material und Methoden}
\chapter{Ergebnisse}
\chapter{Auswertung}
\chapter{Zusammenfassung und Ausblick}


\begin{singlespacing}
\printbibliography
\end{singlespacing}

\appendix %Entfernen/Auskommentieren um Anhang zu deaktivieren
\chapter{Beispiel 1} %Entfernen/Auskommentieren um Anhang zu deaktivieren

\end{document}
